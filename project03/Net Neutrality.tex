\documentclass[11pt]{article}

\begin{document}
\title{COMP 401 - Net Neutrality in the U.S.}
\author{Nicholas Elliot}
\date{April 14, 2015}
\maketitle


\section{Introduction}
\subsection{Definition}
An ``open Internet" is the concept of a worldwide infrastructure that can be browsed freely, fully, and without penalties or regulations.  ``Net neutrality" is the guarantee of an open Internet.

\subsection{Concept}
The first action taken by the United States government involving the \emph{concept} of network neutrality took place June 19, 1934 when the Federal Communications Commissions (FCC) itself was established.  The Communications Act of 1934 was written to prevent communication service providers from discriminating against public use of wire and radio communications \cite{ComAct}.  Simply put: the government agreed that consumers should not be treated differently by communication service providers based on who the user was, who the user was attempting to communicate with, and/or what the user wanted to communicate.

\subsection{History}
January 12, 2003, the term ``net neutrality" was coined by law Professor Tim Wu of Columbia University \cite{ComAct}.  In the late 1990's, there were concerns of the possible subversion of the Internet's ``end-to-end" scheme--where functionality remains in the end hosts.  In response to the possible threats, Wu made an argument that information networks are often more valuable when all content, sites, and platforms are treated equally \cite{Wu}.  Wu has written several papers on this topic where he discusses the benefits to maintaining a neutral information network and suggests methods in which neutrality could be employed within an information network.

\subsection{Law}
In May of 2010, the FCC introduced the first explicit net neutrality protections.  These principles stated that Internet Service Providers (ISP's) ``could not block Websites or impose limits on users" \cite{WH}.  On December 21, 2010, a final version of the rules were passed as the ``Open Internet Rules" which specified the following \cite{FCC}:
\begin{itemize}
\item No Blocking: broadband providers may not block access to legal content, applications, services, or non-harmful devices.
\item No Throttling: broadband providers may not impair or degrade lawful Internet traffic on the basis of content, applications, services, or non-harmful devices.
\item No Paid Prioritization: broadband providers may not favor some lawful Internet traffic over other lawful traffic in exchange for consideration of any kind--in other words, no ``fast lanes." This rule also bans ISP's from prioritizing content and services of their affiliates.
\end{itemize}

In January 2014, the court deemed the FCC's legal framework  to the Open Internet Rules questionable.  Since the FCC had classified broadband Internet access service as an ``information service," rather than a ``telecommunications service," they ``lacked the authority to implement and enforce those rules" \cite{FP}.

In May of 2014, the FCC released a plan that opposed the ``No Paid Prioritization" rule from the Open Internet Rules established in 2010.  This new proposal would have allowed ISP's to create ``pay-to-play" fast lanes. 

On Feb. 4, 2015, after receiving and examining 4 million public comments worth of feedback, the FCC dismissed their original plan to allow paid prioritization in the net \cite{WH}\cite{FP}.

On Feb. 26, 2015, the FCC reclassified Internet broadband as a telecommunication service which allowed them to pass the ``Title II Net Neutrality Rules" which strictly enforce net neutrality rules on Tittle II of the Communications Act--Common Carrier--and reinstated the bans on ``throttling," ``blocking," and ``paid prioritization" \cite{FP}.

\subsection{Violations}
\subsubsection{Madison River Communications Blocks Vonage}
In 2004, Vonage customers who had Madison River Communications as an ISP were restricted from using VoIP service--service which turns a broadband connection into and inexpensive phone line \cite{DD}.  MRC blocked Vonage--a direct competitor--VoIP service to force customers to purchase MRC telephone service.  Vonage filed a complaint with the FCC and, after a short investigation, MRC was required to pay a \$15,000 fine and prohibited from blocking any ports for three years.

\subsubsection{Verizon Blocks Pro-Choice Text Messages}
In 2007, Verizon refused to provide service to abortion rights group NARAL Pro-Choice America, promoting a text message fundraising campaign that would allow supporters to make donations via text. \cite{DD}  Verizon was the only major cellar provider to deny support, stating that the company ``does not accept issue-oriented programs \cite{DD}.  Activist groups argued that Verizon's censorship was politically bias.  In response to this accusation, Verizon quickly retracted their original statement and decided to permit the NARAL test messaging campaign, admitting that they had made a mistake denying the group access in the first place.

\subsubsection{Verizon Blocks Tethering Applications}
In 2008, Verizon blocked and removed third-party tethering applications from the Verizon application store in an attempt to force Verizon customers to pay additional fees to utilize Verizon's own Mobile Broadband Connect tethering service.  After a year-long investigation, Verizon was charged by the FCC and forced to pay a \$1.25 million dollar settlement, eliminate additional charges for Verizon's tethering service, and allow third-party tethering applications to return to the app store.

\subsubsection{AT\&T Blocks Apple FaceTime}
``In 2012, AT\&T blocked Apple's video chat app FaceTime from running on its mobile network unless customers paid extra for a Mobile Shared Data plan" \cite{DD}.  AT\&T explained that the usage of the app was consuming so much bandwidth that the network could not keep up, so they restricted the app usage by only permitting its use by customers who paid for the more expensive data plan, which would reduce strain on the network infrastructure.  Open internet advocacy groups argued that ``blocking FaceTime was an attempt by AT\&T to effectively prohibit the use of an application that would compete with its own voice services" and threatened to file a formal complaint with the FCC \cite{DD}.  Eventually, AT\&T agreed to comply by allowing FaceTime and other video chat applications run on a standard mobile plan by the following year.

\subsubsection{Comcast Blocks BitTorrent}
``In the late 2000's, Comcast began blocking customers from sharing files via BitTorrent because of its heavy bandwidth usage by ``intercepting the data transmitted between the downloader and the file's host and then sending a message to both parties' computers telling them [to] disconnect from each other"" \cite{DD}.  Although BitTorrent is known for the use of illegal file sharing, even legitimate copyright holders were prevented from distributing their content.  Early on, Comcast denied interference with BitTorrent traffic but the FCC launched an investigation and, in mid-2008, ruled that Comcast was was in violation of net neutrality rules \cite{DD}.  Comcast challenged the ruling and, two years later, the court ruled the FCC had ``exceeded its authority when it sanctioned Comcast \cite{DD}.  After their victory, Comcast directed their attention to other congestion-management techniques that did not involve blocking.

\section{Analysis}
\subsection{Economics}
Net neutrality is a necessary component to a strong economy.  An open access Internet makes it possible for small businesses, start-up companies, and entrepreneurs to compete and succeed.  Without a neutral public network, it would be impossible for them to ``launch their businesses, create a market, advertise their products and services, and distribute products to customers" \cite{FP}.  

Net neutrality ensures the Web is a fair playing field.  This is especially important for young or small businesses with limited resources.  Without an open Internet, ISP's, the gatekeepers to the Internet, would ``seize every possible opportunity to profit from that gatekeeper control" \cite{FP}.  ISP's could threaten to throttle or even block business unless they paid a fee.  ISP's could just as easily offer prioritization to businesses that paid for it.

Start-ups and small businesses would not be able to compete with larger corporations that would be able to pay high ISP service fees.  This would increase the risk, if not make it completely impossible, for new and small businesses to launch or survive. In turn, net neutrality fosters job growth, cultivates fair competition, and promotes innovation.

\subsection{Freedom of Speech}
The Internet is a form of telecommunication.  It is unbiased in allowing access to a vast amount of information.  If Web traffic was to be regulated, censored, or blocked, the right to freedom of speech would be denied.  It is contrary to the Constitution to restrict to whom or what a person communicates.

\section{Conclusion}
The Internet is a powerful tool and resource.  It is just as important economically as it is socially.  An information network is most valuable when it is neutral, allowing information to be freely accessed and shared.  Assuming it is legal content, Internet traffic should not be regulated by ISP's.


\begin{thebibliography}{9}
\bibitem{ComAct}
	Communications Act of 1934,
  	\emph{et seq}.
  	(1934). Print.

\bibitem{Wu}
	Wu, Tim.
	``Network Neutrality FAQ."
  	\emph{Network Neutrality FAQ}.
  	N.p., n.d. Web. 14 Apr. 2015.

\bibitem{WH}
	``Net Neutrality: A Free and Open Internet."
  	\emph{The White House}.
  	The White House, n.d. Web. 13 Apr. 2015.

\bibitem{FCC}
	``Open Internet."
  	\emph{Open Internet}.
  	N.p., n.d. Web. 14 Apr. 2015.

\bibitem{FP}
	``Net Neutrality: What You Need to Know Now."
  	\emph{Save the Internet}.
  	Freepress, n.d. Web. 14 Apr. 2015.

\bibitem{DD}
	Sankin, Aaron.
	``The Worst Net Neutrality Violations in History."
  	\emph{The Daily Dot}.
  	The Daily Dot, 21 May 2014. Web. 14 Apr. 2015.
\end{thebibliography}

\end{document}