\documentclass[11pt]{article}

\begin{document}
\title{COMP 401 - Net Neutrality in the U.S.}
\author{Nicholas Elliot}
\date{April 14, 2015}
\maketitle


\section{Introduction}
\subsection{Definition}
An "open Internet" is the concept of a worldwide infrastructure that can be browsed freely, to its depths, without penalties or regulations.  "Net neutrality" is the guarantee of an open Internet.

\subsection{Concept}
The first action taken by the United States government involving the concept of net neutrality took place June 19, 1934 when the Federal Communications Commissions (FCC) itself was established.  The Communications Act of 1934 was written to prevent communication service providers from discriminating against public use of wire and radio communications \cite{ComAct}.  Simply put: the government agreed that consumers should not be treated differently by communication service providers based off of who the user was, who the user was attempting to communicate with, or what the user wanted to communicate.

\subsection{History}
January 12, 2003, the term "net neutrality" was coined by law Professor Tim Wu of Columbia University \cite{ComAct}.  In the late 1990's, there were concerns of the possible subversion of the Internet's "end-to-end" scheme--where functionality remains in the end hosts.  In response to the possible threats, Wu made an argument that information networks are often more valuable when all content, sites, and platforms are treated equally \cite{Wu}.  Wu has written several papers on this topic where he discusses the benefits to maintaining a neutral information network and suggests methods in which neutrality could be employed within an information network.

\subsection{Law}
In May of 2010, the FCC introduced the first explicit net neutrality protections.  These principles stated that Internet Service Providers (ISP's) "could not block Websites or impose limits on users" \cite{WH}.  On December 21, 2010, a final version of the rules were passed as the "Open Internet Rules" which specified the following \cite{FCC}:
\begin{itemize}
\item No Blocking: broadband providers may not block access to legal content, applications, services, or non-harmful devices.
\item No Throttling: broadband providers may not impair or degrade lawful Internet traffic on the basis of content, applications, services, or non-harmful devices.
\item No Paid Prioritization: broadband providers may not favor some lawful Internet traffic over other lawful traffic in exchange for consideration of any kind--in other words, no "fast lanes." This rule also bans ISP's from prioritizing content and services of their affiliates.
\end{itemize}

In January 2014, the court deemed the FCC's legal framework  to the Open Internet Rules questionable.  Since the FCC had classified broadband Internet access service as "information service," rather than "telecommunications service," they "lacked the authority to implement and enforce those rules" \cite{FP}.

In May of 2014, the FCC released a plan that opposed the "No Paid Prioritization" rule from the Open Internet Rules established in 2010.  This new proposal would have allowed ISP's to create "pay-to-play" fast lanes. 

On Feb. 4, 2015, after receiving and examining 4 million public comments worth of feedback, the FCC dismissed their original plan to allow paid prioritization in the net \cite{FP}\cite{WH}.

On Feb. 26, 2015, the FCC reclassified Internet broadband as a telecommunication service which allowed them to pass the "Title II Net Neutrality Rules" which strictly enforce net neutrality rules on Tittle II of the Communications Act--Common Carrier--and reinstated the bans on "throttling," "blocking," and "paid prioritization" \cite{FP}.


\section{Analysis}
\subsection{Economics}
Net neutrality is a necessary component to a strong economy.  An open access Internet makes it possible for small businesses, start-up companies, and entrepreneurs to succeed.  Without a neutral public network, it would be impossible for them to "launch their businesses, create a market, advertise their products and services, and distribute products to customers" \cite{FP}.  

Net neutrality ensures the Web is a fair playing field.  This is especially important for young or small businesses.  Without an open Internet, ISP's--the gatekeepers to the Internet--would "seize every possible opportunity to profit from that gatekeeper control" \cite{FP}.  ISP's could threaten to throttle or even block business unless they paid a fee.  ISP's could just as easily offer prioritization to businesses who paid for it.

Start-ups and small businesses would not be able to compete with larger corporations that would be able to pay off high ISP service fees.  This would increase the risk, if not make it completely impossible, for new and small businesses to launch or survive. In turn, net neutrality fosters job growth, cultivates fair competition, and promotes innovation.

\subsection{Freedom of Speech}
The Internet is a form of telecommunication.  It connects everyone and holds a vast amount of information.  If Web traffic was to be regulated, censored, or blocked, the right to freedom of speech would be denied.  It is against what is said in the constitution to restrict who or what a person communicates with or searches, respectively.

\section{Conclusion}
The Internet is a powerful tool and resource.  It is just as important economically as it is socially, and vice versa.  An information network is most valuable when it is neutral; the most information can be accessed and shared.  Assuming it is legal content, Internet traffic should not be regulated by ISP's.


\begin{thebibliography}{9}
\bibitem{ComAct}
	Communications Act of 1934,
  	\emph{et seq}.
  	(1934). Print.

\bibitem{Wu}
	Wu, Tim.
	"Network Neutrality FAQ."
  	\emph{Network Neutrality FAQ}.
  	N.p., n.d. Web. 14 Apr. 2015.

\bibitem{WH}
	Wu, Tim.
	"Network Neutrality FAQ."
  	\emph{Network Neutrality FAQ}.
  	N.p., n.d. Web. 14 Apr. 2015.

\bibitem{FCC}
	"Open Internet."
  	\emph{Open Internet}.
  	N.p., n.d. Web. 14 Apr. 2015.

\bibitem{FP}
	"Net Neutrality: What You Need to Know Now."
  	\emph{Save the Internet}.
  	Freepress, n.d. Web. 14 Apr. 2015.
\end{thebibliography}

\end{document}